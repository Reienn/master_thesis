\chapter{Zakończenie}
\thispagestyle{firststyle}

W ramach niniejszej pracy wykonano internetową grę dla dzieci z zaburzeniami ze spektrum autyzmu mającą na celu ćwiczenie umiejętności czytania ze zrozumieniem.
Jednym z~założeń projektu było wykorzystanie zalet technologii informacyjno-komunikacyjnych dla edukacji i terapii osób z ASD.
Na podstawie analizy procesu projektowania i implementacji gry przedstawić można spostrzeżenia dotyczące czynników szczególnie obiecujących z perspektywy tworzenia praktycznego narzędzia przeznaczonego dla opisywanej grupy docelowej.
Jeden z nich stanowi zapewnienie środowiska odpowiadającego potrzebom osób z ASD poprzez możliwość ćwiczenia pożądanej umiejętności w kontrolowanych warunkach, służącą wzrostowi pozytywnego nastawienia, motywacji i zaangażowania.
Jako drugi istotny aspekt wymienić należy możliwość precyzyjnego dostosowania interwencji do indywidualnych preferencji i~zdolności użytkowników oraz bieżącego monitorowania uzyskiwanych postępów przez osobę nadzorującą proces edukacyjny.
Ponadto wykonanie gry w formie aplikacji internetowej wiąże się z większą dostępnością i wygodą dla odbiorców, pozwalając na korzystanie z niej na dowolnym komputerze lub tablecie niezależnie od systemu operacyjnego, bez konieczności pobrania i instalacji oprogramowania.

Podczas tworzenia aplikacji zidentyfikowano również trzy główne wyzwania, wymagające wnikliwszej uwagi podczas dalszego rozwoju i ewaluacji gry.
Kluczowym problemem jest zagadnienie trafności, związane z oceną stopnia w jakim zadania dotyczą umiejętności, którą w założeniu mają ćwiczyć.
Trudność w zaprojektowaniu trafnych ćwiczeń wiąże się z brakiem możliwości ścisłego wyodrębnienia czytania ze zrozumieniem od pozostałych umiejętności, jednak konstrukcja zadań powinna uniemożliwiać wykonanie ich w oparciu wyłącznie o inne zdolności poznawcze, takie jak pamięć lub dopasowywanie elementów na podstawie podobieństwa wizualnego.

Równie istotna kwestia dotyczy zakresu generalizacji, a więc przełożenia lepszych wyników osiąganych w grze na postępy również poza tym środowiskiem.
W tym kontekście należy zwrócić uwagę czy zastosowany w aplikacji mechanizm zmienności treści w wystarczającym stopniu minimalizuje ograniczenie efektów gry do ćwiczonego materiału.

Trzecim wartym rozważenia aspektem jest optymalne połączenie funkcji edukacyjnej z~rozrywkową, tak aby gra jednocześnie opierała się na zadaniach ćwiczących czytanie ze zrozumieniem oraz budziła zainteresowanie graczy i zachęcała do wykonywania ćwiczeń.

Gra udostępniona została uczniom i nauczycielom SOSW (Specjalnego Ośrodka Szkolno-Wychowawczego) pn. Centrum Autyzmu i Całościowych Zaburzeń Rozwojowych w Krakowie oraz na forum internetowym Fundacji SYNAPSIS (znajdującym się pod adresem: \url{http://autyzmwpolsce.pl/forum}) celem przetestowania przez użytkowników i uzyskania ich opinii na temat przystępności, użyteczności i atrakcyjności narzędzia.
W~aplikacji, zarówno w~części przeznaczonej dla gracza, jak i w panelu opiekuna, uwzględniono moduł pozwalający użytkownikom na przekazanie informacji zwrotnej poprzez wypełnianie krótkiej ankiety.
Po~uzyskaniu większej liczby ocen możliwa będzie wstępna ewaluacja takich elementów gry, jak interfejs użytkownika.

Aby zweryfikować skuteczność gry w realizowaniu zamierzonego celu edukacyjnego niezbędne jest przeprowadzenie badań, obejmujących porównanie poziomu umiejętności czytania ze zrozumieniem przed i po treningu wykorzystującym grę.
Badania takie pozwoliłyby na odniesienie się do opisanych powyżej kwestii, a także identyfikację innych potencjalnych trudności, umożliwiając właściwą ewaluację w celu udoskonalenia gry pod względem skuteczności.