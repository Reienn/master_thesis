\chapter{Wprowadzenie teoretyczne}
\thispagestyle{firststyle}

\section{Ogólna charakterystyka zaburzeń ze spektrum autyzmu}

    \subsection{Rys historyczny}
    Termin ,,autyzm'' (od gr. \emph{autós} -- ,,sam'') wprowadzony został na początku XX wieku przez psychiatrę Eugena Bleuera w odniesieniu do jednego z zaburzeń występującego w schizofrenii, charakteryzującego się wycofaniem ze świata zewnętrznego (\cite{frith2008autyzm}).
    Określenie to~wykorzystali Leo Kanner i Hans Asperger w opublikowanych niezależnie w 1943 i 1944 roku pracach, analizujących zaburzenie rozwojowe zidentyfikowane przez nich u grupy dzieci.
    Kanner w artykule \emph{Autystyczne zaburzenia kontaktu afektywnego} wprowadził jako nową jednostkę kliniczną ,,autyzm wczesnodziecięcy'' (Kanner i in., \cite*{kanner1943autistic}).
    Opisał przypadki ośmiu chłopców i trzech dziewczynek, u których zaobserwował objawy izolowania się (w odróżnieniu od występującego w schizofrenii wycofania z wcześniej istniejących relacji), zaburzenia mowy (m.in. echolalię, dosłowność), potrzebę niezmienności, stereotypie, bardzo dobrą pamięć mechaniczną.
    Asperger w swojej pracy opisał tzw. ,,psychopatię autystyczną'', charakteryzującą się objawami podobnymi do tych wymienianych przez Kannera, m.in. trudnościami w integracji społecznej, izolowaniem się, niechęcią wobec zmian, stereotypowymi zachowaniami, a~także wąskimi zainteresowaniami i szczególnymi umiejętnościami (\cite{asperger1991autistic}).
    W 1981 roku Lorna Wing nadała tym zaburzeniom nazwę ,,zespołu Aspergera'' (\cite{wing1981asperger}).
    W 1979 roku Wing i Judith Gould wprowadziły określenie ,,kontinuum autystyczne'', zwracając uwagę na różne nasilenie analizowanych zaburzeń (Wing i Gould, \cite*{wing1979severe}).
    
    \subsection{Klasyfikacja i epidemiologia}
    W dwóch pierwszych wydaniach klasyfikacji zaburzeń psychicznych Amerykańskiego Towarzystwa Psychiatrycznego -- DSM-I (Diagnostic and Statistical Manual of Mental Disorders, 1st Edition, \cite{dsm1}) i DSM-II (APA, \cite*{dsm2}) -- autyzm traktowano jako wczesny przejaw psychozy lub schizofrenii dziecięcej, co zrewidowały prace Kolvina i Ruttera (\cite{kolvin1972infantile}, \cite{rutter1972childhood}).
    Klasyfikacja DSM-III (APA, \cite*{dsm3}) wyróżniła autyzm dziecięcy jako osobną jednostkę, zaliczoną do klasy całościowych zaburzeń rozwoju.
    W~uaktualnionym wydaniu DSM-III-R (APA, \cite*{dsm3r}) zmieniono nazwę na zaburzenie autystyczne (kładąc nacisk na podejście rozwojowe) oraz przedstawiono bardziej elastyczne kryteria diagnozy (szesnaście kryteriów podzielonych na trzy kategorie: zaburzenia interakcji społecznych, zaburzenia komunikacji, ograniczony repertuar zachowań i zainteresowań).
    Negatywną konsekwencją zmian było zwiększenie liczby fałszywie pozytywnych diagnoz, a~także wzrastająca niespójność z powstającą wówczas dziesiątą edycją Międzynarodowej Statystycznej Klasyfikacji Chorób i Problemów Zdrowotnych (ICD-10, International Statistical Classification of Diseases and Related Health Problems, 10th Revision, \cite{icd10}).
    Kwestie te uwzględniono w klasyfikacji DSM-IV (APA, \cite*{dsm4}), w której do całościowych zaburzeń rozwojowych obok autyzmu zaliczono również zespół Aspergera, zespół Retta, zespół Hellera (zaburzenia dezintegracyjne) i całościowe zaburzenie rozwoju niezdiagnozowane inaczej.
    W~piątej edycji DSM (APA, \cite*{dsm5}) klasę  całościowych zaburzeń rozwoju zastąpiono pojedynczą kategorią ,,zaburzeń ze spektrum autyzmu'' (ASD, \emph{autism spectrum disorder}), a~także zredukowano kryteria diagnostyczne do dwóch domen -- zaburzeń komunikacji społecznej oraz ograniczonych, powtarzalnych wzorców zachowań i zainteresowań (Maenner i~in., \cite*{maenner2014potential}).
    
    Pierwsze badanie epidemiologiczne, przeprowadzone w 1966 roku przez Victora Lottera, wskazywało na obecność autyzmu u 4,5 na 10 000 przypadków (\cite{lotter1966epidemiology}).
    Obecnie rozpowszechnienie zaburzeń ze spektrum autyzmu w krajach rozwiniętych szacuje się na ok. 1,5\%, średnio czterokrotnie częściej u chłopców niż dziewczynek (Lyall i in., \cite*{lyall2017changing}).
    Tak znaczne zwiększenie szacunków wynika z rozszerzenia kryteriów diagnostycznych i poprawy diagnostyki, nie pozwala więc wnioskować o rzeczywistym wzroście częstości występowania autyzmu w populacji (\cite{frith2008autyzm}).  
    
    \subsection{Symptomy}
    Tradycyjnie za najbardziej charakterystyczne symptomy autyzmu uznaje się tzw. triadę zaburzeń: jakościowe nieprawidłowości w interakcjach społecznych, jakościowe nieprawidłowości w komunikacji oraz ograniczone, powtarzające się zachowania, zainteresowania i aktywności (\cite{frith2008autyzm}).
    Na współwystępowanie zaburzeń w tych trzech obszarach zwróciły uwagę Wing i Gould w badaniu z 1979 roku (Wing i Gould, \cite*{wing1979severe}).
    Symptomy ujawniają się przed 3. rokiem życia (Komender, Jagielska i Bryńska, \cite*{komender2012autyzm}).
    
    Do behawioralnych oznak jakościowych zaburzeń interakcji społecznych należą: obniżona wrażliwość na bodźce społeczne, deficyt wspólnej uwagi, ograniczone posługiwanie się kontaktem wzrokowym i gestykulacją (głównie gesty instrumentalne), trudności w wyrażaniu ekspresji emocjonalnych oraz rozpoznawaniu i reagowaniu na emocje innych (Komender i in., \cite*{komender2012autyzm}).
    Wing i Gould wyróżniły trzy charakterystyczne typy zachowania dzieci z autyzmem w~sytuacjach społecznych: wycofanie, bierność lub specyficzną aktywność (Wing i Gould, \cite*{wing1979severe}).
    
    Zaburzenia komunikacji werbalnej i niewerbalnej obejmują takie symptomy, jak opóźniony lub całkowicie zahamowany rozwój języka przy braku rekompensacji mowy w alternatywny sposób (gestami bądź mimiką), echolalię (natychmiastową lub odroczoną), a także niewystępowanie zróżnicowanej, spontanicznej zabawy w udawanie (Komender, Jagielska i Bryńska, \cite*{komender2012autyzm}).
    
    Ograniczony, repetytywny zestaw czynności przejawia się występowaniem stereotypii ruchowych i perseweracji, zachowywaniem sztywnych procedur i schematów, oporem przed ich zmianą, niestandardowym używaniem zabawek (np. koncentracją na właściwościach zapachowych lub dotykowych) oraz przejawianiem bardzo wąskich zainteresowań (Komender i in. \cite*{komender2012autyzm}).
    
    U osób z zaburzeniami ze spektrum autyzmu często obserwowane są również objawy związane z odbiorem wrażeń zmysłowych -- podwyższona lub obniżona wrażliwość sensoryczna na~niektóre bodźce (Pellicano i Burr, \cite*{pellicano2012world}).
    %Zdolności wysepkowe
    
    \subsection{Teorie poznawcze}
    Wśród poznawczych teorii autyzmu dominują trzy podejścia, skupiające się kolejno na~deficycie teorii umysłu, słabej koherencji centralnej oraz zaburzeniach funkcji wykonawczych (Rajendran i Mitchell, \cite*{rajendran2007cognitive}).
    
    \subsubsection{Deficyt teorii umysłu}
    \label{subsubsection:teoriaumyslu}
    Jedna z głównych hipotez próbujących wyjaśnić zaburzenia ze spektrum autyzmu dotyczy deficytu teorii umysłu. Termin ,,teoria umysłu'' wprowadzili w 1978 roku David Premack i~Guy Woodruff, definiując go jako przypisywanie sobie i innym stanów mentalnych (Premack i Woodruff, \cite*{premack1978does}).
    W 1985 roku Simon Baron-Cohen, Alan Leslie i Uta Frith zaproponowali wyjaśnienie zaburzeń interakcji społecznych i komunikacji charakterystycznych dla autyzmu poprzez deficyt teorii umysłu (Baron-Cohen, Leslie i Frith, \cite*{baron1985does}).
    Wykorzystali stworzoną przez Heinza Wimmera i Josefa Pernera metodę badania rozwoju zdolności rozumienia fałszywych przekonań (Wimmer i Perner, \cite*{wimmer1983beliefs}).
    Na jej podstawie Baron-Cohen opracował test ,,Sally-Anne'', polegający na zaprezentowaniu dziecku przez eksperymentatora scenki z użyciem dwóch lalek.
    Jedna z postaci, Sally, umieszcza szklaną kulkę w koszyku, następnie podczas jej nieobecności Anne przekłada kulkę do pudełka.
    Zadaniem dziecka jest odpowiedź na~pytanie, gdzie Sally będzie szukała swojej kulki.
    Wyniki eksperymentu okazały się zgodne z hipotezą badaczy -- 80\% dzieci z autyzmem udzieliło błędnej odpowiedzi, w odróżnieniu od grupy kontrolnej: dzieci neurotypowych i dzieci z zespołem Downa (Baron-Cohen i in., \cite*{baron1985does}).
    W 1989 roku Baron-Cohen zmodyfikował swoją teorię w celu zwiększenia jej uniwersalności, postulując, że w przypadku dzieci z autyzmem występuje opóźnienie rozwoju teorii umysłu, nie zaś deficyt (\cite{baron1989autistic}).
    Wykorzystał trudniejsze zadanie testujące rozumienie fałszywych przekonań drugiego rzędu, uzyskując 100\% błędnych odpowiedzi u dzieci z autyzmem.
    % Osoby z łagodniejszymi zaburzeniami teorii umysłu mogą poprzez kompensacyjne uczenie się do pewnego stopnia przyswoić sobie umiejętność przypisywania stanów mentalnych (\cite{frith2008autyzm}). 
    
    \subsubsection{Słaba centralna koherencja}
    \label{subsubsection:teoriakoherencji}
    Kolejna teoria poznawcza -- teoria słabej centralnej koherencji -- skupia się na wyjaśnieniu różnic w procesach przetwarzania informacji u osób z zaburzeniami ze spektrum autyzmu w~porównaniu do osób neurotypowych (Happ{\'e} i Frith, \cite*{happe2006weak}).
    W 1989 roku Frith zdefiniowała centralną koherencję jako skłonność do uspójniania i generalizowania przetwarzanych informacji (\cite{frith2008autyzm}).
    Zdaniem Frith w przypadku osób z autyzmem tendencja ta jest osłabiona, co przejawia się w skoncentrowaniu na detalach oraz w trudnościach w uchwyceniu całości informacji i umieszczeniu jej w szerszym kontekście.
    Potwierdzenie tej hipotezy stanowić mogą badania zdolności wzrokowo-przestrzennych, np. wyższe wyniki uzyskane przez dzieci z~autyzmem w teście ukrytych figur (Shah i Frith, \cite*{shah1983islet}) oraz w zadaniu Wzory z klocków -- podteście Skali Inteligencji Wechslera (Shah i Frith, \cite*{shah1993autistic}).
    Bazując na teorii słabej centralnej koherencji Francesca Happ{\'e} zweryfikowała hipotezę niższej podatności dzieci z~autyzmem na~złudzenia wzrokowe (m.in. złudzenie Ponza, Ebbinghausa), uzyskując rezultat zgodny z~przewidywaniami (\cite{happe1996studying}), jednak badania Ropar i Mitchella nie zreplikowały tych wyników (Ropar i Mitchell, \cite*{ropar1999individuals}).
    Słaba centralna koherencja może tłumaczyć również zaburzenia przebiegu wyższopoziomowych procesów, takich jak przetwarzanie informacji zależnych od kontekstu semantycznego (\cite{frith2008autyzm}).
    Według badań wykorzystujących zdania z~homografami (słowami o identycznej pisowni, różniącymi się wymową i znaczeniem) osoby z~autyzmem rzadziej uwzględniają kontekst wpływający na znaczenie wyrazu (Frith i~Snowling, \cite*{frith1983reading}, \cite{happe1997central}, Jolliffe i Baron-Cohen, \cite*{jolliffe1999test}).  
    
    \subsubsection{Zaburzenia funkcji wykonawczych}
    Teoria zaburzeń funkcji wykonawczych związana jest z badaniami pacjentów z uszkodzeniami mózgu (płatów czołowych) objawiającymi się symptomami podobnymi do tych występujących u osób z zaburzeniami ze spektrum autyzmu, takimi jak perseweracje, trudności w~przełączaniu się między zadaniami i hamowaniu dominujących reakcji (\cite{frith2008autyzm}, \cite{russell1997autism}).
    Sally Ozonoff i wsp. definiują funkcje wykonawcze jako zbiór zdolności umożliwiających utrzymanie właściwego sposobu rozwiązywania problemu ukierunkowanego na~osiągnięcie przyszłego celu, zawierający planowanie, kontrolę impulsów, hamowanie dominujących reakcji, metodyczne przeszukiwanie, elastyczność myślenia i działania (Ozonoff, Pennington i Rogers, \cite*{ozonoff1991executive}).
    Elisabeth Hill podzieliła badania nad funkcjami wykonawczymi u~osób z autyzmem na następujące grupy: planowanie, elastyczność umysłową, hamowanie, generatywność i samomonitorowanie (\cite{hill2004evaluating}).
    Pierwsze cztery z powyższych zdolności badano m.in. z wykorzystaniem klasycznych testów, odpowiednio: zadań typu Wieża Hanoi (Ozonoff i in., \cite*{ozonoff1991executive}, Bennetto, Pennington i Rogers, \cite*{bennetto1996intact}, Ozonoff i Jensen, \cite*{ozonoff1999brief}), testu sortowania kart z Wisconsin (Grant i Berg, \cite*{grant1948behavioral}; Ozonoff i in., \cite*{ozonoff1991executive}, Ozonoff i Jensen, \cite*{ozonoff1999brief}), testu Stroopa (\cite{stroop1935studies}; Ozonoff i Jensen, \cite*{ozonoff1999brief}) i testu płynności werbalnej (\cite{turner1999generating}), ostatnią badano nowszymi, eksperymentalnymi testami. 
    Niespójne wyniki badań wskazują na~ograniczenia teorii dysfunkcji wykonawczych, deficyty te nie są bowiem uniwersalne ani unikalne dla autyzmu, występując również w takich zaburzeniach, jak ADHD czy zespół Tourette'a (\cite{hill2004evaluating}).
    
    \subsection{Etiologia i podłoże neuronalne}
    Historycznie przyczyn autyzmu poszukiwano w czynnikach psychogennych, głównie w~zaburzonej relacji z rodzicami, co nie znalazło jednak potwierdzenia w badaniach (Komender, Jagielska i Bryńska, \cite*{komender2012autyzm}). 
    
    Do środowiskowych czynników łączonych ze zwiększonym ryzykiem autyzmu zalicza się infekcje wirusowe i autoimmunologiczne oraz powikłania w okresie ciąży i porodu (szczególnie zakażenie różyczką w pierwszym trymestrze), zaburzenia metaboliczne, a także zaawansowany wiek rodziców (Komender i in., \cite*{komender2012autyzm}, \cite{frith2008autyzm}).
    Szeroko komentowana teoria potencjalnego wpływu szczepionek (m.in. trójskładnikowej szczepionki przeciw odrze, śwince i różyczce) na wystąpienie objawów autystycznych została zweryfikowana w licznych badaniach, które nie wykazały takiej zależności (\cite{frith2008autyzm}, Taylor, Swerdfeger i Eslick, \cite*{taylor2014vaccines}).
    
    Wśród badań potwierdzających znaczenie czynników genetycznych jako przyczyny zaburzeń ze spektrum autyzmu można wyróżnić trzy główne grupy: badania bliźniąt, badania osób spokrewnionych oraz badania rzadkich chorób genetycznych ze współwystępującymi objawami autyzmu (\cite{geschwind2011genetics}).
    Badania wskazują na wysoką (nawet powyżej 90\%) zgodność występowania ASD u bliźniąt monozygotycznych (Folstein i Rutter, \cite*{folstein1977infantile}, Tick, Bolton, Happ{\'e}, Rutter i Rijsdijk, \cite*{tick2016heritability}).
    Urodzenie dziecka z zaburzeniami autystycznymi zwiększa prawdopodobieństwo ASD u kolejnego potomka do kilku--kilkunastu procent (Ozonoff i in., \cite*{ozonoff2011recurrence}).
    Do chorób uwarunkowanych genetycznie, w przypadku których często występują objawy charakterystyczne dla autyzmu, należą m.in. stwardnienie guzowate, zespół łamliwego chromosomu X, nerwiakowłókniakowatość, zespół Retta (Jeste i Geschwind, \cite*{jeste2014disentangling}), zespół Smitha-Lemliego-Opitza (Sikora, Pettit-Kekel, Penfield, Merkens i Steiner, \cite*{sikora2006near}).
    Liczne badania, identyfikujące geny związane z powstawaniem autyzmu (np. geny z rodziny SHANK i NRXN) wskazują na dużą heterogeniczność etiologiczną (Yuen i in., \cite*{yuen2017whole}), zaś specyficzną przyczynę genetyczną udaje się ustalić u około 15\% (Carter i Scherer, \cite*{carter2013autism}). 
    
    Dzięki wykorzystaniu metod neuroobrazowania możliwe jest badanie neuronalnego podłoża zaburzeń ze spektrum autyzmu, związanego z anomaliami anatomicznymi i funkcjonalnymi.
    Do najlepiej udokumentowanych różnic w stosunku do typowego rozwoju należy przyspieszony przyrost objętości mózgu w początkowym etapie życia, skutkujący średnio o 5--10\% większą objętością mózgu u dzieci w wieku od 18 miesięcy do 4 lat (Amaral, Schumann i Nordahl, \cite*{amaral2008neuroanatomy}).
    W późniejszym okresie następuje jednak przedwczesne wstrzymanie wzrostu (a nawet anormalny spadek) objętości mózgu i liczby neuronów (Courchesne, Campbell i Solso, \cite*{courchesne2011brain}).
    Nadmierne zwiększenie obserwuje się szczególnie w przypadku płata czołowego i skroniowego oraz ciała migdałowatego (Courchesne i in., \cite*{courchesne2011brain}).
    Specyficzne anomalie zlokalizowano również w móżdżku, gdzie obserwuje się zmniejszoną liczbę komórek Purkinjego (Fatemi i in., \cite*{fatemi2012consensus}).
    Dane uzyskane z wykorzystaniem obrazowania tensora dyfuzji (niższa wartość anizotropii frakcyjnej) świadczą o obniżonej integralność istoty białej (Travers i in., \cite*{travers2012diffusion}).
    Analiza mikroskopowej struktury mózgu wskazuje na występowanie anomalii w organizacji kolumn neuronalnych w korze mózgowej (Casanova i in., \cite*{casanova2006minicolumnar}).
    Badania wskazują również na nadmiar połączeń synaptycznych, wynikający z zaburzenia mechanizmów eliminowania synaps w~procesie neuronalnej autofagii (Tang i in., \cite*{tang2014loss}). 

\section{Trudności w rozumieniu języka u osób z ASD}

    % \subsection{Rozumienie języka}
    % % teorie poznawcze, neuronalne korelaty
    
    % Wśród teorii rozumienia języka, tłumaczących mechanizmy syntaktycznego i semantycznego przetwarzania zdań, wyróżnić można model ślepej uliczki (\emph{garden-path}), model spełniania ograniczeń (\emph{constraint-satisfaction}) oraz teorię dostatecznych reprezentacji (\emph{good-enough representations}, \cite{ferreira2002good}).
    
    % Według modelu ślepej uliczki przetwarzanie zdań ma charakter sekwencyjny (\cite{ferreira2002good}, \cite{frazier1979comprehending}).
    % Interpretacja będąca efektem analizy syntaktycznej zostaje następnie zweryfikowana w oparciu o aspekt semantyczny i kontekst.
    % Model spełniania ograniczeń
    % wieloznaczność
    
    % Złożona zdolność, jaką stanowi rozumienie języka, zarówno mówionego jak i pisanego
    % Rozumienie języka, zarówno mówionego, jak i pisanego, stanowi złożone zadanie, na które składa się wiele różnych umiejętności i procesów poznawczych (\cite{cain2008children}). % zmodyfikować
    
    % Tak zwany prosty model czytania (\emph{Simple View of Reading}) wyróżnia dwa komponenty: dekodowanie i rozumienie języka (\cite{gough1986decoding}).
    
    % Wśród wyjaśnień mechanizmu wizualnego rozpoznawania słów, stanowiącego podstawowy komponent czytania, wyróżnić można dwa główne podejścia -- teorie podwójnej drogi oraz teorie koneksjonistyczne (\cite{coltheart2005modeling}).
    % Teorii podwójnej drogi wyróżniają dwie strategie czytanie wyrazów: leksykalną i nieleksykalną (\cite{coltheart2006dual}).
    % Pierwsza z nich (wzrokowa, logograficzna) polega na bezpośrednim przejściu od graficznej formy wyrazu do jego znaczenia oraz wymowy i stosowana jest w przypadku znanych słów (\cite{sochacka2004rozwoj}).
    % W drugiej (fonologicznej) rozpoznawanie znaczenia następuje w oparciu o brzmienie wyrazu, uzyskiwane poprzez przekształcanie poszczególnych grafemów (liter) na fonemy (głoski).
    % Według teorii koneksjonistycznych reprezentacje wyrazów mają charakter rozproszony i są przetwarzane równolegle (\cite{plaut2005connectionist}).
    
    % Z formalnego punktu widzenia rozumienie zdania można określić jako rozumienie wszystkich jego aspektów (słów, konstrukcji gramatycznych, terminów deiktycznych, logicznych związków między składowymi zdania złożonego), zarówno występujących w konkretnym kontekście, jak i w oderwaniu od kontekstu (\cite{lord1985autism}).
    % Zazwyczaj rozumienie każdego z elementów wypowiedzi nie jest jednak niezbędne do uchwycenia jej sensu, ponieważ sytuacja dostarcza dodatkowych informacji, wpływających na interpretację znaczenia.
    
    % Rozumienie tekstu pisanego można rozpatrywać jako przetwarzanie na kilku poziomach (\cite{kintsch2005comprehension}).
    % Poziom lingwistyczny obejmuje procesy percepcyjne (dekodowanie graficznych symboli), identyfikowanie słów oraz ich funkcji syntaktycznej.
    % Kolejny poziom stanowi analiza semantyczna, w ramach której budowana jest mikrostruktura tekstu 
    
    \subsection{Specyfika zachowań językowych}
    Jeden z symptomów zaburzeń ze spektrum autyzmu stanowią trudności w sferze językowej, obejmujące rozumienie języka oraz specyficzne różnice w mowie (\cite{frith2008autyzm}).
    W~przypadku większości dzieci z ASD rozwój językowy jest spowolniony w porównaniu do dzieci neurotypowych, zaś mowa czynna pojawia się później, u części dzieci ulega regresowi lub nie występuje wcale (Tager-Flusberg, Paul, Lord i in., \cite*{tager2005language}).
    
    Język osób z zaburzeniami ze spektrum autyzmu cechuje się używaniem wyrazów o idiosynkratycznych znaczeniach (mających sens tylko dla mówiącego), wynikających z jednostkowych skojarzeń.
    Badania wskazują, że dzieci z autyzmem nabywają znaczenia słów głównie w oparciu o skojarzenie dźwięku z obserwowanym przez siebie obiektem, nie zaś z obiektem wskazywanym przez kierunek patrzenia mówiącego, co może prowadzić do przyswojenia błędnego skojarzenia (Baron-Cohen, Baldwin i Crowson, \cite*{baron1997children}).
    
    Zjawiskiem często obserwowanym u dzieci z autyzmem jest mowa echolaliczna, polegająca na dokładnym powtarzaniu zasłyszanej wypowiedzi (Tager-Flusberg i in., \cite*{tager2005language}).
    Echolalia występuje w formie natychmiastowej lub odroczonej, która może pełnić funkcję komunikatu wyrażającego chęć odtworzenia sytuacji skojarzonej z daną frazą.
    Powtarzanie wypowiedzi nie zawsze jednak służy komunikacji, czasem stanowiąc wyłącznie rodzaj stereotypii (\cite{frith2008autyzm}).
    
    Wśród zachowań językowych spotykanych u dzieci z ASD wymienia się także nieprawidłowe stosowanie zaimków, obejmujące m.in. zamienianie zaimków osobowych ,,ja'' i ,,ty'' (Tager-Flusberg i in., \cite*{tager2005language}).
    Jedno z wyjaśnień tłumaczy tego rodzaju błędy przypisując je zjawisku echolalii (\cite{frith2008autyzm}).
    Bardziej złożone wyjaśnienie łączy je z ogólnymi trudnościami w~deiktycznym aspekcie języka, dotyczącym terminów, których znaczenie jest względne i zmienia się zależnie od perspektywy mówiącego i słuchacza (np. ,,tutaj'' i ,,tam'').
    
    Wypowiedzi osób z autyzmem często charakteryzują się także specyficzną prozodią (Tager-Flusberg i in., \cite*{tager2005language}).
    Nietypowości te mogą obejmować intonację (monotonną lub śpiewną), tempo wypowiedzi (zbyt wolne lub zbyt szybkie), wysokość i natężenie głosu oraz gwałtowne zmiany sposobu mówienia (\cite{frith2008autyzm}).
    
    % Badania Rity {\v{C}}eponien{\.e} i wsp. z wykorzystaniem analizy potencjałów zdarzeniowych w paradygmacie bodźca odstającego (\emph{oddball}) wykazały różnice w orientowaniu uwagi (wyrażonym przez potencjał P3a) przez dzieci z autyzmem na mowę oraz na bodźce dźwiękowe niebędące mową (\cite{vceponiene2003speech}).
 
    % Postrzeganie wszystkiego w czarno-białych kategoriach
    % Precyzja w definiowaniu terminów, składni i fonologii wypowiedzi
    % Zaburzona kompetencja konwersacyjna
    %     Badania Chrisitane Baltaxe - niemieckie nastolatki mylą formę grzecznościową ze zwykłym zaimkiem osobowym (nie biorą pod uwagę ról społecznych); problemy związane z kolejnością zabierania głosu, odróżnianiem nowych informacji od starych (nie zawsze sygnalizowanie zmiany tematu)
    
    %     Problemy z przekazywaniem istotnych (nowych) informacji
    %     Problemy z intencjonalną komunikacją

    \subsection{Rozumienie języka}
    \label{subsection:rozumienie}
    Badania analizujące zaburzenia językowe spotykane u dzieci z autyzmem wskazują, że~rozwój zdolności fonologicznych i syntaktycznych ma przebieg zbliżony (choć spowolniony) do obserwowanego u dzieci neurotypowych oraz dzieci z innymi zaburzeniami, zaś specyficzne trudności dotyczą przede wszystkim aspektu semantycznego i pragmatycznego (\cite{tager1981nature}).
    Trudności w rozumieniu języka wynikać mogą z ograniczonej zdolności integrowania treści wypowiedzi z posiadaną wiedzą lub z deficytów w zakresie kompetencji społecznych (Tager-Flusberg i in., \cite*{tager2005language}).
    
    Do specyfiki języka osób z ASD należą deficyty w zakresie kompetencji pragmatycznej, przejawiające się dosłownym rozumieniem pośrednich aktów mowy (np. prośby wyrażonej poprzez pytanie), bez uwzględnienia intencji mówiącego (\cite{frith2008autyzm}).
    Literalne odbieranie wypowiedzi w oderwaniu od kontekstu, mogącego wpłynąć na jej znaczenie, prowadzi też do trudności w interpretacji ironii oraz humoru.
    
    U dzieci z autyzmem zauważa się częstsze wykorzystywanie informacji syntaktycznych (np. kolejności słów w zdaniu) niż strategii semantycznej (prawdopodobieństwo sytuacji) w~zadaniach sprawdzających rozumienie zdań (\cite{tager1981sentence}, Paul, Fischer i Cohen, \cite*{paul1988brief}).
    W zadaniach obejmujących odpamiętywanie (swobodne lub ze wskazówką) listy słów powiązanych lub niepowiązanych semantycznie dzieci z autyzmem osiągały istotnie niższe wyniki w~warunku swobodnego odpamiętywania powiązanych słów, natomiast w warunku ze~wskazówką nie występowały różnice w porównaniu z grupą kontrolą (\cite{tager1991semantic}).
    
    Badania przeprowadzone przez Josepha McCleery'ego i wsp. z wykorzystaniem analizy potencjałów zdarzeniowych wykazały różnice między werbalnym i niewerbalnym przetwarzaniem semantycznym u dzieci z ASD (McCleery i in., \cite*{mccleery2010neural}).
    Podczas pomiaru EEG uczestnikom prezentowano zgodne lub niezgodne pary bodźców wzrokowych (zdjęć) i bodźców słuchowych (słów lub odgłosów, np. dźwięku uruchamianego silnika samochodu).
    U dzieci z grupy kontrolnej w warunkach niespójnych (zarówno w przypadku słów, jak i odgłosów) zaobserwowano potencjał N400 (interpretowany jako reakcja na niezgodność semantyczną), w~odróżnieniu od dzieci z ASD, u których efekt ten wystąpił jedynie w warunku z niezgodnym odgłosem.
    
    % zdania na zapamiętywanie - rzadziej grupują semantycznie
    % trudność w operowaniu symbolami, wpływająca na zdolność formowania wyższopoziomowych abstrakcji i rozumienia złożonych relacji między różnymi kategoriami
    % przypominanie par skojarzeń - nie wykorzystują relacji
    % idiomy
    % general ‘semantic’ processing deficit. For example, Peters (1986) proposed that autism might be associated with impairments in lexical-semantics due to the importance of social interaction and social learning for the development of the language system.
    
    % \cite{frith2008autyzm}
    %   Dzieci z ASD lubią czytać, ale nie po to, by zrozumieć tekst
    %   Problemy z zadaniami na uzupełnianie brakujących słów, wskazywanie niepasujących słów
    % Kristina Scheuffgen - Skupienie bardziej na poszczególnych słowach niż na całej historii
    % \cite{jones2009reading}, \cite{ricketts2013reading}, \cite{lord1985autism}

    % U części osób hiperleksja
    % Jeśli chodzi o czytanie, u osób z ASD często obserwuje się większe trudności w rozumieniu w porównaniu do rozpoznawania słów (\cite{ricketts2013reading}).
    % U części dzieci występuje hiperleksja.
    
    Badania metodą funkcjonalnego rezonansu magnetycznego przeprowadzone przez Marcela Justa i wsp. wykazały istotne różnice w aktywacji struktur kluczowych dla języka u osób z autyzmem w porównaniu z grupą kontrolną (Just, Cherkassky, Keller i Minshew, \cite*{just2004cortical}).
    Uczestnicy (siedemnaście wysokofunkcjonujących osób z autyzmem oraz siedemnaście osób neurotypowych o zbliżonym ilorazie inteligencji werbalnej) wykonywali zadania polegające na czytaniu zdań w stronie czynnej lub biernej oraz identyfikowaniu wykonawcy lub odbiorcy czynności.
    Wyniki potwierdziły hipotezę przewidującą, że u osób z autyzmem w związku z bardziej rozwiniętą zdolnością przetwarzania słów wystąpi większa aktywacja ośrodka Wernickego, zaś z~uwagi na ograniczoną integrację składowych zdania w strukturę syntaktyczną i semantyczną -- mniejsza aktywacja ośrodka Broki oraz niższa funkcjonalna łączność obszarów korowych.
    W badaniu przeprowadzonym przez Gordona Harrisa i wsp. uzyskano podobne zróżnicowanie w aktywacji obszarów Broki i Wernickego, u osób z ASD zaobserwowano ponadto mniejszą różnicę aktywacji podczas przetwarzania konkretnych oraz abstrakcyjnych słów (Harris i in., \cite*{harris2006brain}).
    
    Trudności w rozumieniu obejmują zarówno język mówiony, jak i pisany (Ricketts, Jones, Happ{\'e} i Charman, \cite*{ricketts2013reading}).
    Dekodowanie, pierwszy komponent składowy procesu czytania w~ujęciu tzw. prostego modelu czytania (\emph{Simple View of Reading}, Gough i Tunmer, \cite*{gough1986decoding}), jest niezaburzony -- u osób z ASD przeważnie nie obserwuje się deficytów zdolności wizualnego rozpoznawania słów (Randi, Newman i Grigorenko, \cite*{randi2010teaching}).
    Drugi komponent, rozumienie języka, zazwyczaj cechuje się nieproporcjonalnie niższym poziomem funkcjonowania, niż komponent pierwszy (Huemer i Mann, \cite*{huemer2010comprehensive}).
    Taki profil zaburzeń czytania określany bywa jako hiperleksja i stanowi wzorzec odwrotny niż w dysleksji, w której występuje deficyt zdolności dekodowania (\cite{aaron2012dyslexia}).

\section{ICT w terapii osób z zaburzeniami ze spektrum autyzmu}

    \subsection{Potencjał narzędzi ICT w pracy z osobami z ASD}
    \label{subsection:ict}
    W ciągu ostatnich lat zaobserwować można dynamiczny wzrost zainteresowania wykorzystaniem technologii informacyjno-komunikacyjnych (ICT, \emph{information and communication technology}) w edukacji, między innymi w kontekście osób ze specjalnymi potrzebami edukacyjnymi (Konstantinidis i in., \cite*{konstantinidis2009information}, Grynszpan, Weiss, Perez-Diaz i Gal, \cite*{grynszpan2014innovative}).
    Wiele badań zwraca uwagę na pozytywny odbiór nowych technologii przez osoby z zaburzeniami ze~spektrum autyzmu (Goldsmith i LeBlanc, \cite*{goldsmith2004use}).
    Urządzenia elektroniczne i programy komputerowe zapewniają użytkownikom bezpieczne środowisko, w którym mogą rozwijać i doskonalić pożądaną umiejętność (Grynszpan i in., \cite*{grynszpan2014innovative}).
    Gwarantują je kontrolowane warunki pozwalające na zminimalizowanie dystraktorów oraz jasno zdefiniowane i ustrukturyzowane zadania (Ramdoss i in., \cite*{ramdoss2011use}).
    Środowisko wirtualne cechuje się niższą złożonością niż sytuacje w~świecie rzeczywistym, redukując lęk społeczny i umożliwiając naukę w indywidualnym tempie (Putnam i Chong, \cite*{putnam2008software}).
    Interakcja z komputerem przebiega w oparciu o przewidywalne reguły oraz dostarcza szybkich, powtarzalnych efektów.
    W związku z tym forma zadania sama w sobie może być dla ucznia atrakcyjnym bodźcem, stanowiąc pozytywne wzmocnienie.
    
    Z perspektywy nauczycieli i terapeutów do zalet metod opartych na wykorzystaniu ICT należą możliwość precyzyjnego dostosowania interwencji do indywidualnych potrzeb, poziomu umiejętności i preferencji uczniów, kontrolowania jej przebiegu, a także łatwiejszego monitorowania postępów (Boucenna i in., \cite*{boucenna2014interactive}).
    
    %Wśród krytycznych głosów na temat wykorzystania ICT w pracy z osobami z zaburzeniami ze spektrum autyzmu pojawia się argument zwracający uwagę na ryzyko pogłębienia się społecznej izolacji osób z ASD poprzez ograniczenie kontaktów społecznych na rzecz interakcji z komputerem (\cite{rajendran2013virtual}).
    % kontrargumenty + źródła krytyki
    
    W ramach rozwiązań przeznaczonych dla osób z zaburzeniami ze spektrum autyzmu, opartych na technologiach informacyjno-komunikacyjnych, można wyróżnić narzędzia diagnostyczne, edukacyjne oraz ułatwiające komunikację i funkcjonowanie w życiu codziennym (Landowska, Kołakowska, Anzulewicz, Jarmołkowicz i Rewera, \cite*{landowska2014technologieWEdukacji}).
    
    \subsection{Narzędzia wspomagające diagnozę}
    Rozwój możliwości technologicznych współczesnych komputerów stwarza obiecujące perspektywy dla diagnozy zaburzeń ze spektrum autyzmu.
    Przykład takiego zastosowania stanowi aplikacja przeznaczona na tablety, wykorzystująca dane rejestrowane przez sensory, w które wyposażone jest urządzenie (Anzulewicz, Sobota i Delafield-Butt, \cite*{anzulewicz2016toward}).
    Analiza z użyciem uczenia maszynowego pozwala na identyfikację wzorców ruchowych charakterystycznych dla zaburzeń ze spektrum autyzmu.
    
    Do technologii analizowanych pod względem użyteczności we wczesnej diagnozie ASD należy okulografia, umożliwiająca m.in. pomiar czasu skupiania wzroku na bodźcach społecznych (Falck-Ytter, B{\"o}lte i Gredeb{\"a}ck, \cite*{falck2013eye}).
    Badania wykazują, że osoby z autyzmem krócej skupiają wzrok na twarzach, szczególnie na okolicach oczu innych osób (Falck-Ytter i in., \cite*{falck2013eye}, Jones, Carr i Klin, \cite*{jones2008absence}, Jones i Klin, \cite*{jones2013attention}).
    
    Zastosowanie w rozpoznawaniu behawioralnych symptomów zaburzeń ze spektrum autyzmu mogą mieć również systemy komputerowego przetwarzania obrazu (Hashemi i in., \cite*{hashemi2012computer}).
    Automatyczne narzędzia pozwalają na znaczne przyspieszenie procesu analizy nagrań wideo z~obserwacji zachowania dziecka oraz nie wymagają zaangażowania specjalisty.
    
    \subsection{Narzędzia wspomagające funkcjonowanie}
    ICT wykorzystuje się również do tworzenia narzędzi asystujących, wspomagających osoby z autyzmem w życiu codziennym (Boucenna i in., \cite*{boucenna2014interactive}). % poprawa jakości życia?
    Obejmują one rozwiązania bazujące na alternatywnych i wspomagających metodach komunikacji (AAC, \emph{augmentative and alternative communication}), dostępne w formie dedykowanych urządzeń (np. komunikator GoTalk) lub oprogramowania na komputery osobiste, w tym urządzenia mobilne (Shane i in., \cite*{shane2012applying}).
    Do~kategorii tej należy aplikacja MÓWik na tablety i smartfony z systemem Android, pozwalająca na konstruowanie wypowiedzi poprzez wybór symboli widocznych na ekranie (\cite{markuc2014wspomaganie}).
    Aplikacja zawiera rozszerzalną bazę 9700 symboli oraz wbudowany syntezator mowy Ivona, wykorzystywany do odczytywania słów i zdań.
    % I Can Word It Too (\cite{hetzroni2004effects}), Proloquo2go
    % organizacja
    % bolte2010can
    
    \subsection{Narzędzia wspomagające edukację i terapię}
    Istotny obszar zastosowania nowych technologii w pracy z osobami z zaburzeniami ze spektrum autyzmu stanowią rozwiązania przeznaczone do celów edukacyjnych i terapeutycznych (Boucenna i in., \cite*{boucenna2014interactive}).
    W ramach tych narzędzi wyróżnić można programy komputerowe, w~tym aplikacje przeznaczone na urządzenia mobilne, oraz interaktywne roboty.
    Zorientowane są one przede wszystkim na rozwijanie takich sfer jak interakcje społeczne, komunikacja, wspólna uwaga, rozpoznawanie emocji.
    
    % Wykorzystanie robotów
    Znaczący wkład w wykorzystanie technologii w terapii mają osiągnięcia robotyki społecznej, zajmującej się tworzeniem robotów zdolnych do interakcji społecznych i komunikacji z~ludźmi (\cite{breazeal2004designing}).
    Wykorzystaniu robotów w terapii dzieci z autyzmem poświęcony jest m.in. projekt AURORA (\emph{AUtonomous RObotic platform as a Remedial tool for children with Autism}), zapoczątkowany w 1998 przez Kerstin Dautenhahn (\cite{dautenhahn1999robots}).
    Terapia z udziałem robotów umożliwia zwiększenie zainteresowania i zaangażowania uczestników oraz ułatwia trening imitacji i wspólnej uwagi (Boucenna i in., \cite*{boucenna2014interactive}).
    W porównaniu do metod opartych na środowisku wirtualnym skuteczność terapii z wykorzystaniem robotów jest w~mniejszym stopniu ograniczona do kontekstu interwencji terapeutycznej.
    Pod względem wyglądu zewnętrznego roboty stosowane w interwencjach terapeutycznych cechują się zróżnicowanym stopniem antropomorfizacji, obejmując roboty humanoidalne, roboty przypominające zwierzęta oraz roboty nieinspirowane biologicznie (Scassellati, Admoni i Matari{\'c}, \cite*{scassellati2012robots}).
    Jednym z humanoidalnych robotów wykorzystywanych w terapii osób z ASD jest KASPAR o~wyglądzie małego chłopca (Dautenhahn i in., \cite*{dautenhahn2009kaspar}).
    Robot posiada znacznie uproszczoną mimikę, która ma służyć zredukowaniu złożoności interakcji (Robins, Dautenhahn i Dickerson, \cite*{robins2009isolation}).
    % NAO, Leka, Paro?
    
    %Gry poważne w terapii dzieci z ASD
    Programy komputerowe wykorzystywane w terapii ASD często należą do kategorii tzw. ,,gier poważnych'' (\emph{serious games}).
    Podstawy tej idei sformalizował Clark C. Abt, określając powyższym terminem gry, które posiadają jasno zdefiniowany i zaplanowany cel edukacyjny, a funkcja rozrywkowa nie jest ich podstawowym przeznaczeniem (\cite{abt1987serious}). 
    Poprzez medium jakie stanowi gra możliwe jest rozwijanie zróżnicowanych zdolności, nabywanie wiedzy, wspomaganie terapii i rehabilitacji (Michael i Chen, \cite*{michael2005serious}). % immersja  breuer2010so
    Do zalet takiego rozwiązania należą zwiększenie motywacji użytkowników, dostarczanie natychmiastowej informacji zwrotnej oraz łatwość dostosowania do wybranej tematyki (Ritterfeld, Cody i Vorderer, \cite*{ritterfeld2009serious}).
    
    Grossard i wsp. analizują ponad trzydzieści gier poważnych ukierunkowanych na trening umiejętności społecznych, klasyfikując je na dwie kategorie -- gry dotyczące usprawniania rozpoznawania emocji oraz gry zaprojektowane w celu rozwijania ogólnych umiejętności społecznych, m.in. interakcji i współpracy (Grossard i in., \cite*{grossard2017serious}).
    Autorzy pozytywnie oceniają perspektywy wykorzystania ICT w terapii osób z zaburzeniami ze spektrum autyzmu.
    Zwracają jednocześnie uwagę na ograniczenia metodologiczne i potrzebę wnikliwszej empirycznej walidacji skuteczności wielu z proponowanych rozwiązań, jak również większej liczby gier dostosowanych do osób z autyzmem niskofunkcjonującym.
    
    % ECHOES
    Przykładem gry poważnej zaprojektowanej w celu nabywania i rozwijania zdolności komunikacji społecznej przez dzieci z ASD jest gra ECHOES (Bernardini, Porayska-Pomsta i Smith, \cite*{bernardini2014echoes}).
    Składa się ona z ćwiczeń doskonalących wspólną uwagę oraz posługiwanie się symbolami werbalnymi i niewerbalnymi, podzielonych na dwie kategorie -- ćwiczenia nastawione na~cel oraz na współpracę.
    Za pośrednictwem dużego ekranu dotykowego oraz okulografu gracz wchodzi w interakcje z autonomicznym bohaterem Andym zamieszkującym wirtualny ogród sensoryczny.
    Andy nawiązuje kontakt z graczem, reaguje na jego zachowanie i wspiera w wykonywaniu zadań.
    % Autorzy przeprowadzili wstępną analizę efektów sześciotygodniowego treningu z udziałem 19 dzieci w wieku od 4 do 14 lat mierząc liczbę prób inicjowania i podtrzymywania interakcji z wirtualnym bohaterem oraz z człowiekiem.
    % Nie zaobserwowali istotnego transferu wzrostu zaangażowania w interakcje społeczne ze środowiska gry do sytuacji rzeczywistych, jednak zwrócili uwagę na . % wyrzucić lub dodać jakieś ale

    % JeStiMulE
    Znaczną grupę gier poważnych przeznaczonych dla osób z zaburzeniami ze spektrum autyzmu stanowią gry trenujące rozpoznawanie emocji, takie jak gra JeStiMulE, której zakres obejmuje rozpoznawanie ekspresji mimicznych, gestów oraz emocji w kontekście sytuacji społecznych (Serret i in., \cite*{serret2014facing}).
    Gra wykorzystuje stymulację multisensoryczną -- bodźce wzrokowe, słuchowe oraz dotykowe, generowane za pomocą gamepada (urządzenia do sterowania) wyposażonego w stymulatory wibro-dotykowe.
    Autorzy zweryfikowali efektywność gry w badaniu z udziałem grupy 33 osób z ASD, zróżnicowanej pod względem wieku (od 6 do 17 lat) oraz poziomu funkcjonowania.
    Badanie wykazało istotną poprawę w rozpoznawaniu emocji po czterotygodniowym treningu z wykorzystaniem gry, zarówno w zadaniach z wirtualnymi postaciami z gry jak i w zadaniach ze zdjęciami rzeczywistych osób.
    