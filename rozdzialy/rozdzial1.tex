\chapter{Wprowadzenie teoretyczne}
\thispagestyle{firststyle}

\section{Ogólna charakterystyka zaburzeń ze spektrum autyzmu}

    \subsection{Rys historyczny}
    Termin ,,autyzm'' (od gr. \emph{autós} -- ,,sam'') wprowadzony został na początku XX wieku przez psychiatrę Eugena Bleuera w odniesieniu do jednego z zaburzeń występującego w schizofrenii, charakteryzującego się wycofaniem ze świata zewnętrznego (\cite{frith2008autyzm}).
    Określenie to wykorzystali Leo Kanner i Hans Asperger w opublikowanych niezależnie w 1943 i 1944 roku pracach, analizujących zaburzenie rozwojowe zidentyfikowane przez nich u grupy dzieci.
    Kanner w artykule \emph{Autystyczne zaburzenia kontaktu afektywnego} wprowadził jako nową jednostkę kliniczną ,,autyzm wczesnodziecięcy'' (\cite{kanner1943autistic}).
    Opisał przypadki ośmiu chłopców i trzech dziewczynek, u których zaobserwował objawy izolowania się (w odróżnieniu do występującego w schizofrenii wycofania z wcześniej istniejących relacji), zaburzenia mowy (m.in. echolalię, dosłowność), potrzebę niezmienności, stereotypie, bardzo dobrą pamięć mechaniczną.
    Asperger w swojej pracy opisał tzw. ,,psychopatię autystyczną'', charakteryzującą się objawami podobnymi do tych wymienianych przez Kannera, m.in. trudnościami w integracji społecznej, izolowaniem się, niechęcią wobec zmian, stereotypowymi zachowaniami, a także wąskimi zainteresowaniami i szczególnymi umiejętnościami (\cite{asperger1991autistic}).
    W 1981 roku Lorna Wing nadała tym zaburzeniom nazwę ,,zespołu Aspergera'' (\cite{wing1981asperger}).
    W 1979 roku Wing i Judith Gould wprowadziły określenie ,,kontinuum autystyczne'', zwracając uwagę na różne nasilenie analizowanych zaburzeń (\cite{wing1979severe}).
    
    % Zaburzenia ze spektrum autyzmu (ASD, \emph{autism spectrum disorder})
    
    \subsection{Klasyfikacja i epidemiologia}
    W dwóch pierwszych wydaniach klasyfikacji zaburzeń psychicznych Amerykańskiego Towarzystwa Psychiatrycznego (DSM-I i DSM-II) autyzm traktowano jako wczesny przejaw psychozy lub schizofrenii dziecięcej, co zrewidowały prace Kolvina i Ruttera (\cite{kolvin1972infantile}, \cite{rutter1972childhood}).
    Klasyfikacja DSM-III wydana w 1980 roku wyróżniła autyzm dziecięcy jako osobną jednostkę, zaliczoną do klasy całościowych zaburzeń rozwoju (\cite{volkmar2014kanner}).
    W uaktualnionym wydaniu DSM-III-R z 1987 roku zmieniono nazwę na zaburzenie autystyczne (kładąc nacisk na podejście rozwojowe) oraz przedstawiono bardziej elastyczne kryteria diagnozy (szesnaście kryteriów podzielonych na trzy kategorie: zaburzenia interakcji społecznych, zaburzenia komunikacji, ograniczony repertuar zachowań i zainteresowań).
    Negatywną konsekwencją zmian było zwiększenie liczby fałszywie pozytywnych diagnoz, a także wzrastająca niespójność z powstającą wówczas dziesiątą edycją Międzynarodowej Statystycznej Klasyfikacji Chorób i Problemów Zdrowotnych (ICD-10).
    Kwestie te uwzględniono w opublikowanej w 1994 roku klasyfikacji DSM-IV, w której do całościowych zaburzeń rozwojowych obok autyzmu zaliczono również zespół Aspergera, zespół Retta, zespół Hellera (zaburzenia dezintegracyjne) i całościowe zaburzenie rozwoju niezdiagnozowane inaczej (\cite{volkmar2014kanner}).
    W wydanej w 2013 roku piątej edycji DSM klasę całościowych zaburzeń rozwoju zastąpiono pojedynczą kategorią ,,zaburzeń ze spektrum autyzmu'' (ASD, \emph{autism spectrum disorder}), a także zredukowano kryteria diagnostyczne do dwóch domen -- zaburzeń komunikacji społecznej oraz ograniczonych, powtarzalnych wzorców zachowań i zainteresowań (\cite{maenner2014potential}).
    
    Obecnie rozpowszechnienie zaburzeń ze spektrum autyzmu w krajach rozwiniętych szacuje się na ok. 1,5\% (\cite{lyall2017changing}).
    Występują one średnio czterokrotnie częściej u chłopców niż dziewczynek (\cite{lyall2017changing}).
    
    \subsection{Symptomy}
    Diagnoza autyzmu opiera się na współwystępowaniu trzech kryteriów: zaburzenia interakcji społecznych, zaburzenia komunikacji oraz ograniczonych, powtarzalnych czynności i zachowań (\cite{frith2008autyzm}).
    
    \subsection{Teorie poznawcze}
    Wśród poznawczych teorii autyzmu dominują trzy podejścia, skupiające się kolejno na deficycie teorii umysłu, słabej koherencji centralnej oraz zaburzeniach funkcji wykonawczych (\cite{rajendran2007cognitive}).
    
    Jedna z głównych hipotez próbujących wyjaśnić zaburzenia ze spektrum autyzmu dotyczy deficytu teorii umysłu. Termin ,,teoria umysłu'' wprowadzili w 1978 roku David Premack i Guy Woodruff, definiując go jako przypisywanie sobie i innym stanów mentalnych (\cite{premack1978does}).
    W 1985 roku Simon Baron-Cohen, Alan Leslie i Uta Frith zaproponowali wyjaśnienie zaburzeń interakcji społecznych i komunikacji charakterystycznych dla autyzmu poprzez deficyt teorii umysłu (\cite{baron1985does}).
    Wykorzystali stworzoną przez Heinza Wimmera i Josefa Pernera metodę badania rozwoju zdolności rozumienia fałszywych przekonań (\cite{wimmer1983beliefs}).
    Na jej podstawie Baron-Cohen opracował test ,,Sally-Anne'', polegający na zaprezentowaniu dziecku przez eksperymentatora scenki z użyciem dwóch lalek.
    Jedna z postaci, Sally, umieszcza szklaną kulkę w koszyku, następnie podczas jej nieobecności Anne przekłada kulkę do pudełka.
    Zadaniem dziecka jest odpowiedź na pytanie, gdzie Sally będzie szukała swojej kulki.
    Wyniki eksperymentu okazały się zgodne z hipotezą badaczy -- 80\% dzieci z autyzmem udzieliła błędnej odpowiedzi, w odróżnieniu od grupy kontrolnej: dzieci neurotypowych i dzieci z zespołem Downa (\cite{baron1985does}).
    W 1989 roku Baron-Cohen zmodyfikował swoją teorię w celu zwiększenia jej uniwersalności, postulując, że w przypadku dzieci z autyzmem występuje opóźnienie rozwoju teorii umysłu, nie zaś deficyt (\cite{baron1989autistic}).
    Wykorzystał trudniejsze zadanie testujące rozumienie fałszywych przekonań drugiego rzędu, uzyskując 100\% błędnych odpowiedzi u dzieci z autyzmem.
    
    Kolejna teoria poznawcza -- teoria słabej centralnej koherencji -- skupia się na wyjaśnieniu różnic w procesach przetwarzania informacji u osób ze spektrum autyzmu w porównaniu do osób neurotypowych (\cite{happe2006weak}).
    W 1989 roku Frith zdefiniowała centralną koherencję jako skłonność do uspójniania i generalizowania przetwarzanych informacji (\cite{frith2008autyzm}). 
    Zdaniem Frith w przypadku osób z autyzmem tendencja ta jest osłabiona, co przejawia się w skoncentrowaniu na detalach oraz w trudnościach w uchwyceniu całości informacji i umieszczeniu jej w szerszym kontekście.
    Potwierdzenie tej hipotezy stanowić mogą badania zdolności wzrokowo-przestrzennych, np. wyższe wyniki uzyskane przez dzieci z autyzmem w teście ukrytych figur (\cite{shah1983islet}) oraz w zadaniu Wzory z klocków -- podteście Skali Inteligencji Wechslera (\cite{shah1993autistic}).
    Bazując na teorii słabej centralnej koherencji Francesca Happ{\'e} zweryfikowała hipotezę niższej podatności dzieci z autyzmem na złudzenia wzrokowe (m.in. złudzenie Ponza, Ebbinghausa), uzyskując rezultat zgodny z przewidywaniami (\cite{happe1996studying}), jednak badania Ropar i Mitchella nie zreplikowały tych wyników (\cite{ropar1999individuals}).
    Słaba centralna koherencja może tłumaczyć również wyższopoziomowe procesy, takie jak przetwarzanie informacji zależnych od kontekstu semantycznego (\cite{frith2008autyzm}).
    Według badań wykorzystujących zdania z homografami (słowami o identycznej pisowni, różniącymi się wymową i znaczeniem) osoby z autyzmem rzadziej uwzględniają kontekst wpływający na znaczenie wyrazu (\cite{frith1983reading}, \cite{happe1997central}, \cite{jolliffe1999test}).  
    
    Teoria zaburzeń funkcji wykonawczych związana jest z badaniami pacjentów z uszkodzeniami mózgu (płatów czołowych) objawiającymi się symptomami podobnymi do tych występujących u osób ze spektrum autyzmu, takimi jak perseweracje, trudności w przełączaniu się między zadaniami i hamowaniu dominujących reakcji (\cite{frith2008autyzm}, \cite{russell1997autism}).
    Sally Ozonoff i wsp. definiują funkcję wykonawczą jako zdolność utrzymania właściwego sposobu rozwiązywania problemu ukierunkowanego na osiągnięcie przyszłego celu, zawierającą planowanie, kontrolę impulsów, hamowanie dominujących reakcji, metodyczne przeszukiwanie, elastyczność myślenia i działania (\cite{ozonoff1991executive}).
    Elisabeth Hill podzieliła badania nad funkcjami wykonawczymi u osób z autyzmem na następujące grupy: planowanie, elastyczność umysłową, wyhamowanie, generatywność i samomonitorowanie (\cite{hill2004evaluating}).
    Pierwsze cztery z powyższych zdolności badano m.in. z wykorzystaniem klasycznych testów, odpowiednio: zadań typu Wieża Hanoi (\cite{ozonoff1991executive}, \cite{bennetto1996intact}, \cite{ozonoff1999brief}), testu sortowania kart z Wisconsin (\cite{grant1948behavioral}; \cite{ozonoff1991executive}, \cite{ozonoff1999brief}), testu Stroopa (\cite{stroop1935studies}; \cite{ozonoff1999brief}) i testu płynności werbalnej (\cite{turner1999generating}), ostatnią badano nowszymi, eksperymentalnymi testami. 
    Niespójne wyniki badań wskazują na ograniczenia teorii dysfunkcji wykonawczych, deficyty te nie są bowiem uniwersalne ani unikalne dla spektrum autyzmu, występując również w takich zaburzeniach, jak ADHD czy zespół Tourette'a (\cite{hill2004evaluating}).
    
    \subsection{Etiologia}
    %\subsection{Neuronalne podstawy autyzmu}

\section{Trudności w czytaniu ze zrozumieniem u osób ze spektrum autyzmu}

    \subsection{Czytanie ze zrozumieniem}
     Rozumienie języka, zarówno mówionego, jak i pisanego, stanowi złożone zadanie, na które składa się wiele różnych umiejętności i procesów poznawczych (\cite{cain2008children}). Tak zwany prosty model czytania (\emph{Simple View of Reading}, \cite{hoover1990simple}) wyróżnia dwa komponenty: dekodowanie i rozumienie języka.
     
    \subsection{Trudności językowe u osób z autyzmem}
    

\section{Technologie informacyjno-komunikacyjne w terapii dzieci ze spektrum autyzmu}