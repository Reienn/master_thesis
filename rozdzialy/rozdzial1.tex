\chapter{Wprowadzenie teoretyczne}
\thispagestyle{firststyle}

\section{Ogólna charakterystyka zaburzeń ze spektrum autyzmu}

    \subsection{Rys historyczny}
    Termin ,,autyzm'' (od gr. \emph{autos} -- ,,sam'') wprowadzony został na początku XX wieku przez psychiatrę Eugena Bleuera w odniesieniu do jednego z zaburzeń występującego w schizofrenii, charakteryzującego się wycofaniem ze świata zewnętrznego (\cite{frith2008autyzm}).
    Określenie to wykorzystali Leo Kanner i Hans Asperger w opublikowanych niezależnie w 1943 i 1944 roku pracach, zawierających opisy dzieci o podobnych zaburzeniach (\cite{kanner1943autistic}, \cite{asperger_1991}).
    % Zaburzenia ze spektrum autyzmu (ASD, \emph{autism spectrum disorder})
    
    \subsection{Epidemiologia}
    Obecnie szacuje się, że rozpowszechnienie występowania zaburzeń ze spektrum autyzmu
    
    \subsection{Klasyfikacja}
    % DSM
    
    \subsection{Kryteria}
    Diagnoza autyzmu opiera się na współwystępowaniu trzech kryteriów: zaburzenia interakcji społecznych, zaburzenia komunikacji oraz ograniczonych, powtarzalnych czynności i zachowań (\cite{frith2008autyzm}).
    
    \subsection{Etiologia}
    % Przedstawienie hipotez (deficyt teorii umysłu, słaba koherencja centralna, zaburzenia funkcji wykonawczych)
    
    \subsection{Neuronalne podstawy autyzmu}

\section{Trudności w czytaniu ze zrozumieniem u osób ze spektrum autyzmu}

Rozumienie języka, zarówno mówionego, jak i pisanego, stanowi złożone zadanie, na które składa się wiele różnych umiejętności i procesów poznawczych (\cite{cain2008children}). Tak zwany prosty model czytania (Simple View of Reading, \cite{Hoover1990}) wyróżnia dwa komponenty: dekodowanie i rozumienie języka. 

\section{Technologie informacyjno-komunikacyjne w terapii dzieci ze spektrum autyzmu}