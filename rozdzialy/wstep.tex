\chapter{Wstęp}
\thispagestyle{firststyle}

Przedmiot niniejszej pracy stanowi zaprojektowanie i wykonanie gry komputerowej dla dzieci z zaburzeniami ze spektrum autyzmu (ASD, \emph{autism spectrum disorder}) mającej wspomagać ćwiczenie czytania ze zrozumieniem, umożliwiającej monitorowanie postępów gracza i~personalizację treści gry.
ASD jest zaburzeniem rozwojowym, przejawiającym się przede wszystkim trudnościami w obszarach interakcji społecznych i komunikacji oraz tendencją do~powtarzalnych zachowań i ograniczonych zainteresowań (\cite{frith2008autyzm}).
Do sfer, w obrębie których u osób z ASD zwykle obserwuje się deficyty, należą umiejętności językowe.

Specyfika zaburzeń ze spektrum autyzmu zazwyczaj wiąże się z pozytywnym nastawieniem do~technologii, zapewniającej bezpieczne i ustrukturyzowane środowisko oraz kontrolowane warunki, dlatego interwencje edukacyjne i terapeutyczne przeznaczone dla osób z ASD często opierają się się na zastosowaniu programów komputerowych (Grynszpan, Weiss, Perez-Diaz i~Gal, \cite*{grynszpan2014innovative}).
Opisywane w niniejszej pracy narzędzie zdecydowano się stworzyć w formie tzw. ,,gry poważnej'' (\emph{serious game}), wykorzystując element rozrywkowy w celu zwiększenia zainteresowania i motywacji użytkowników.
Tematykę gry postanowiono osadzić w realiach detektywistycznych, formułując zadania jako rozwiązywanie kolejnych śledztw, związanych ze skradzionymi przedmiotami.

Wśród gier edukacyjnych dla dzieci z zaburzeniami ze spektrum autyzmu zaobserwować można dominację pozycji dotyczących doskonalenia kompetencji społecznych (Zakari, Ma i~Simmons, \cite*{zakari2014review}, Grossard i in., \cite*{grossard2017serious}), w~związku z czym projektowaną grę postanowiono poświęcić rozwijaniu sfery językowej.
Przedmiot gry zawężono do umiejętności rozumienia języka, stanowiącej dla osób z ASD szczególną trudność (\cite{tager1981nature}).

Aby opracować praktyczne narzędzie dydaktyczne, pozwalające na obserwację dokonywanych przez gracza postępów oraz dostosowanie zadań do jego indywidualnych potrzeb i~preferencji zdecydowano się rozbudować grę o panel kontrolny przeznaczony dla nauczyciela, terapeuty lub rodzica.
W tym celu stworzono system zapisywania bieżących wyników gracza, prezentowanych w panelu kontrolnym w postaci statystyk obrazujących poziom wykonywania poszczególnych zadań na przestrzeni czasu.
Założeniem gry jest nauka czytania ze~zrozumieniem poprzez wielokrotne powtarzanie zadań.
Aby uniknąć ryzyka ograniczenia kolejnych prób do~odpamiętywania przykładów bez ich zrozumienia wprowadzono mechanizm zmienności treści gry, polegający na każdorazowym generowaniu poszczególnych ćwiczeń na~podstawie podanego materiału.
W panelu kontrolnym udostępniono opcję modyfikowania domyślnego zestawu treści, umożliwiającą dostosowanie trudności ćwiczeń oraz urozmaicenie rozgrywki.

W obliczu wzrastającej popularności aplikacji internetowych zdecydowano się wykonać grę w tej właśnie formie, mając na celu zwiększenie jej dostępności na różnych platformach (komputerach i tabletach niezależnie od systemu operacyjnego), wyeliminowanie konieczności pobrania i instalacji oraz ułatwienie konserwacji oprogramowania.
Projektując stronę graficzną oraz interfejs użytkownika zwracano uwagę na dostosowanie gry do potrzeb grupy docelowej, poprzez takie aspekty jak przystępna nawigacja, czytelność, rozmieszczenie elementów na ekranie, dobór kolorów oraz styl grafiki.

Pierwsza część pracy stanowi wprowadzenie teoretyczne, zawierające ogólny opis zaburzeń autystycznych, charakterystykę trudności językowych oraz argumenty za wykorzystaniem narzędzi komputerowych w pracy z osobami z ASD.
W kolejnych dwóch rozdziałach przedstawiono odpowiednio: projekt gry, w tym jej koncepcję, tematykę i treść (strukturę poszczególnych ćwiczeń) oraz implementację (opis wykorzystanej technologii i funkcjonalności aplikacji).

