\documentclass[a4paper,12pt]{report}
\usepackage[utf8]{inputenc}
\usepackage{polski}
\usepackage[T1]{fontenc}
\usepackage{mathptmx}
\usepackage{indentfirst}
\linespread{1.5}
\usepackage{tabularx}
\usepackage{graphicx}
\usepackage{subcaption}
\graphicspath{ {obrazy/} }
\usepackage[all]{nowidow}

\usepackage[backend=biber, style=apa, citestyle=apa]{biblatex}
\DeclareLanguageMapping{english}{english-apa}
\addbibresource{bibliografia.bib}
\usepackage{hyperref}

\usepackage{anysize}
\marginsize{3.5cm}{1.6cm}{2.5cm}{2.5cm}

\usepackage{fancyhdr}
\pagestyle{fancy}
\fancyhf{}
\rhead{\thepage}
\lhead{\nouppercase{\itshape\leftmark}}

\fancypagestyle{firststyle}
{
   \fancyhf{}
   \rhead{\thepage}
   \renewcommand\headrulewidth{0pt}
}

\newcommand{\authorname}{Anna Jankowicz}
\newcommand{\supname}{dr Magdalena Senderecka}
\newcommand{\titlepl}{Technologie informacyjno-komunikacyjne\\w terapii dzieci z zaburzeniami ze spektrum autyzmu: gra wspomagająca rozwój umiejętności czytania ze zrozumieniem}
\newcommand{\titleen}{Information and Communication Technology in Therapy for Children with Autism Spectrum Disorder: Game Improving Reading Comprehension}


\title{\titlepl}
\author{\authorname}
\date{2019}

\begin{document}

\pagenumbering{roman}

\thispagestyle{empty}

{\centering\linespread{1.05}
    \textbf{
    {\Large Uniwersytet Jagielloński\\}
    {\large Wydział Filozoficzny\\
      \uppercase{Instytut Filozofii}\\}
      \textbf{Studia stacjonarne\\
      Kierunek Kognitywistyka}\\
    }
    \vspace{0.5cm}    
    {\large{Praca magisterska\\}}
}

\vspace{1cm}
\begin{minipage}[t]{0.4\textwidth}
    \begin{flushleft} \large Nr albumu: 1111251\\ \end{flushleft}
\end{minipage}
\begin{minipage}[t]{0.4\textwidth}
    \begin{flushright} \large Ocena:\\ \end{flushright}
\end{minipage}

\vspace{0.3cm}
{\centering\LARGE{\authorname\\}}
\vspace{0.5cm}
{\centering\linespread{1.15}\LARGE{TECHNOLOGIE INFORMACYJNO-KOMUNIKACYJNE W~TERAPII DZIECI Z ZABURZENIAMI ZE~SPEKTRUM AUTYZMU: GRA WSPOMAGAJĄCA ROZWÓJ UMIEJĘTNOŚCI CZYTANIA ZE ZROZUMIENIEM\\}}

\vspace{2.5cm}
\begin{minipage}[t]{0.4\textwidth}
    \begin{flushleft} \large
    % \emph{Autor:}\\ \authorname
    \end{flushleft}
\end{minipage}
\begin{minipage}[t]{0.5\textwidth}
    \begin{flushright} \large
    Promotor pracy magisterskiej:\\ \supname
    \end{flushright}
\end{minipage}
\vspace{2cm}

{\centering\small{Opracowano zgodnie z obowiązującymi przepisami o prawie autorskim i prawach pokrewnych.}\\ \vspace{0.25cm}\Large{Kraków 2019\\}}


\newpage
\setcounter{page}{1}
\thispagestyle{firststyle}
\vspace*{10cm}
\emph{\\\authorname\\ Numer albumu: 1111251\\}

Oświadczam, że przedstawioną do oceny pracę magisterską, zatytułowaną
,,Technologie informacyjno-komunikacyjne w terapii dzieci z zaburzeniami ze spektrum autyzmu: gra wspomagająca rozwój umiejętności czytania ze zrozumieniem'', wykonałam osobiście i po raz pierwszy poddaję ocenie.

% \addcontentsline{toc}{chapter}{Abstrakt}
\thispagestyle{firststyle}
{\centering\large
    % \vspace{0.5cm}
    % Uniwersytet Jagielloński\\
    {\Huge\emph{Abstrakt}\\}
    \vspace{0.5cm}
    % Wydział Filozoficzny\\Instytut Filozofii\\
    % \vspace{1cm}
    % \textbf{Technologie informacyjno-komunikacyjne w terapii dzieci\\z zaburzeniami ze spektrum autyzmu: gra wspomagająca rozwój umiejętności czytania ze zrozumieniem\\}
    % \vspace{0.5cm}
    % \authorname\\
    % \vspace{0.5cm}
}

\sloppy
Technologie informacyjno-komunikacyjne stanowią narzędzie często wykorzystywane w~edukacji i terapii osób z zaburzeniami ze spektrum autyzmu, z uwagi na łatwość zapewnienia przewidywalnego środowiska i nauki w kontrolowanych warunkach.
Jednym z obszarów, w~ramach których u osób z ASD można zaobserwować trudności, są kompetencje językowe, w~szczególności rozumienie języka.
Niniejsza praca zawiera opis projektu i implementacji gry komputerowej dla dzieci z ASD wspomagającej ćwiczenie umiejętności czytania ze zrozumieniem, pozwalającej na bieżące monitorowanie postępów gracza oraz zindywidualizowanie treści zadań.

\vspace{1cm}
\noindent\textbf{Słowa kluczowe:}
zaburzenia ze spektrum autyzmu, technologie informacyjno-komunikacyjne, czytanie ze zrozumieniem

% \addcontentsline{toc}{chapter}{Abstract}
% \thispagestyle{firststyle}
{\centering\large
    % \vspace{0.5cm}
    % Jagiellonian University\\
    {\Huge\emph{Abstract}\\}
    \vspace{0.5cm}
    % Faculty of Philosophy\\Institute of Philosophy\\
    % \vspace{1cm}
    % \textbf{Information and Communication Technology in Therapy for Children with Autism Spectrum Disorder: Game Improving Reading Comprehension\\}
    % \vspace{0.5cm}
    % \authorname\\
    % \vspace{0.5cm}
}

Information and communication technology is often used in education and therapy for people with autism spectrum disorder due to the possibility of providing predictable environment and learning in controlled conditions.
One of the areas in which people with ASD may have difficulties are language skills, especially language comprehension.
This thesis contains a~description of the project and the implementation of a computer game for children with ASD improving reading comprehension, that allows for ongoing monitoring of the progress of a~player and individualisation of the content of exercises.

\vspace{1cm}
\noindent\textbf{Keywords:}
autism spectrum disorder, information and communication technology, reading comprehension
\addcontentsline{toc}{chapter}{Abstract}
% \thispagestyle{firststyle}
{\centering\large
    % \vspace{0.5cm}
    % Jagiellonian University\\
    {\Huge\emph{Abstract}\\}
    \vspace{0.5cm}
    % Faculty of Philosophy\\Institute of Philosophy\\
    % \vspace{1cm}
    % \textbf{Information and Communication Technology in Therapy for Children with Autism Spectrum Disorder: Game Improving Reading Comprehension\\}
    % \vspace{0.5cm}
    % \authorname\\
    % \vspace{0.5cm}
}

This thesis contains a description of the project and the implementation of a computer game for children with autism spectrum disorder improving reading comprehension, that allows for ongoing monitoring of the progress of a player and individualising the content of the game.
Language skills, especially language comprehension, are one of the areas in which people with ASD have difficulties.
Information and communication technology is often used in education and therapy for people with ASD due to the possibility of providing predictable environment and learning in controlled conditions.

\vspace{1cm}
\noindent\textbf{Keywords:}
autism spectrum disorder, information and communication technology, reading comprehension

\tableofcontents
\thispagestyle{firststyle}
 
\listoftables
\thispagestyle{firststyle}
\begingroup
\let\clearpage\relax
\listoffigures
\endgroup

\chapter{Wstęp}
\thispagestyle{firststyle}
\pagenumbering{arabic}

\section{Cel pracy}

\chapter{Wprowadzenie teoretyczne}
\thispagestyle{firststyle}

\section{Ogólna charakterystyka zaburzeń ze spektrum autyzmu}

    \subsection{Rys historyczny}
    Termin ,,autyzm'' (od gr. \emph{autós} -- ,,sam'') wprowadzony został na początku XX wieku przez psychiatrę Eugena Bleuera w odniesieniu do jednego z zaburzeń występującego w schizofrenii, charakteryzującego się wycofaniem ze świata zewnętrznego (\cite{frith2008autyzm}).
    Określenie to wykorzystali Leo Kanner i Hans Asperger w opublikowanych niezależnie w 1943 i 1944 roku pracach, analizujących zaburzenie rozwojowe zidentyfikowane przez nich u grupy dzieci.
    Kanner w artykule \emph{Autystyczne zaburzenia kontaktu afektywnego} wprowadził jako nową jednostkę kliniczną ,,autyzm wczesnodziecięcy'' (\cite{kanner1943autistic}).
    Opisał przypadki ośmiu chłopców i trzech dziewczynek, u których zaobserwował objawy izolowania się (w odróżnieniu do występującego w schizofrenii wycofania z wcześniej istniejących relacji), zaburzenia mowy (m.in. echolalię, dosłowność), potrzebę niezmienności, stereotypie, bardzo dobrą pamięć mechaniczną.
    Asperger w swojej pracy opisał tzw. ,,psychopatię autystyczną'', charakteryzującą się objawami podobnymi do tych wymienianych przez Kannera, m.in. trudnościami w integracji społecznej, izolowaniem się, niechęcią wobec zmian, stereotypowymi zachowaniami, a także wąskimi zainteresowaniami i szczególnymi umiejętnościami (\cite{asperger1991autistic}).
    W 1981 roku Lorna Wing nadała tym zaburzeniom nazwę ,,zespołu Aspergera'' (\cite{wing1981asperger}).
    W 1979 roku Wing i Judith Gould wprowadziły określenie ,,kontinuum autystyczne'', zwracając uwagę na różne nasilenie analizowanych zaburzeń (\cite{wing1979severe}).
    
    % Zaburzenia ze spektrum autyzmu (ASD, \emph{autism spectrum disorder})
    
    \subsection{Klasyfikacja i epidemiologia}
    W dwóch pierwszych wydaniach klasyfikacji zaburzeń psychicznych Amerykańskiego Towarzystwa Psychiatrycznego (DSM-I i DSM-II) autyzm traktowano jako wczesny przejaw psychozy lub schizofrenii dziecięcej, co zrewidowały prace Kolvina i Ruttera (\cite{kolvin1972infantile}, \cite{rutter1972childhood}).
    Klasyfikacja DSM-III wydana w 1980 roku wyróżniła autyzm dziecięcy jako osobną jednostkę, zaliczoną do klasy całościowych zaburzeń rozwoju (\cite{volkmar2014kanner}).
    W uaktualnionym wydaniu DSM-III-R z 1987 roku zmieniono nazwę na zaburzenie autystyczne (kładąc nacisk na podejście rozwojowe) oraz przedstawiono bardziej elastyczne kryteria diagnozy (szesnaście kryteriów podzielonych na trzy kategorie: zaburzenia interakcji społecznych, zaburzenia komunikacji, ograniczony repertuar zachowań i zainteresowań).
    Negatywną konsekwencją zmian było zwiększenie liczby fałszywie pozytywnych diagnoz, a także wzrastająca niespójność z powstającą wówczas dziesiątą edycją Międzynarodowej Statystycznej Klasyfikacji Chorób i Problemów Zdrowotnych (ICD-10).
    Kwestie te uwzględniono w opublikowanej w 1994 roku klasyfikacji DSM-IV, w której do całościowych zaburzeń rozwojowych obok autyzmu zaliczono również zespół Aspergera, zespół Retta, zespół Hellera (zaburzenia dezintegracyjne) i całościowe zaburzenie rozwoju niezdiagnozowane inaczej (\cite{volkmar2014kanner}).
    W wydanej w 2013 roku piątej edycji DSM klasę całościowych zaburzeń rozwoju zastąpiono pojedynczą kategorią ,,zaburzeń ze spektrum autyzmu'' (ASD, \emph{autism spectrum disorder}), a także zredukowano kryteria diagnostyczne do dwóch domen -- zaburzeń komunikacji społecznej oraz ograniczonych, powtarzalnych wzorców zachowań i zainteresowań (\cite{maenner2014potential}).
    
    Obecnie rozpowszechnienie zaburzeń ze spektrum autyzmu w krajach rozwiniętych szacuje się na ok. 1,5\%  (\cite{lyall2017changing}).
    Występują one średnio czterokrotnie częściej u chłopców niż dziewczynek (\cite{lyall2017changing}).
    
    \subsection{Kryteria}
    Diagnoza autyzmu opiera się na współwystępowaniu trzech kryteriów: zaburzenia interakcji społecznych, zaburzenia komunikacji oraz ograniczonych, powtarzalnych czynności i zachowań (\cite{frith2008autyzm}).
    
    \subsection{Teorie poznawcze}
    Wśród poznawczych teorii autyzmu dominują trzy podejścia, skupiające się na deficycie teorii umysłu, słabej koherencji centralnej oraz zaburzeniach funkcji wykonawczych (\cite{rajendran2007cognitive}).
    
    Jedna z głównych hipotez próbujących wyjaśnić zaburzenia ze spektrum autyzmu dotyczy deficytu teorii umysłu. Termin ,,teoria umysłu'' wprowadzili w 1978 roku David Premack i Guy Woodruff, definiując go jako przypisywanie sobie i innym stanów mentalnych (\cite{premack1978does}).
    W 1985 roku Simon Baron-Cohen, Alan Leslie i Uta Frith zaproponowali wyjaśnienie zaburzeń interakcji społecznych i komunikacji charakterystycznych dla autyzmu poprzez deficyt teorii umysłu (\cite{baron1985does}).
    Wykorzystali stworzoną przez Heinza Wimmera i Josefa Pernera metodę badania rozwoju zdolności rozumienia fałszywych przekonań (\cite{wimmer1983beliefs}).
    Na jej podstawie Baron-Cohen opracował test ,,Sally-Anne'', polegający na zaprezentowaniu dziecku przez eksperymentatora scenki z użyciem dwóch lalek.
    Jedna z postaci, Sally, umieszcza szklaną kulkę w koszyku, następnie podczas jej nieobecności Anne przekłada kulkę do pudełka.
    Zadaniem dziecka jest odpowiedź na pytanie, gdzie Sally będzie szukała swojej kulki.
    Wyniki eksperymentu okazały się zgodne z hipotezą badaczy -- 80\% dzieci z autyzmem udzieliła błędnej odpowiedzi, w odróżnieniu od grupy kontrolnej: dzieci neurotypowych i dzieci z zespołem Downa (\cite{baron1985does}).
    W 1989 roku Baron-Cohen zmodyfikował swoją teorię w celu zwiększenia jej uniwersalności, postulując opóźnienie rozwoju, a nie deficyt, teorii umysłu u dzieci z autyzmem (\cite{baron1989autistic}).
    Wykorzystał trudniejsze zadanie testujące rozumienie fałszywych przekonań drugiego rzędu, uzyskując 100\% błędnych odpowiedzi u dzieci z autyzmem.
    
    
    
    \subsection{Etiologia}
    
    \subsection{Neuronalne podstawy autyzmu}

\section{Trudności w czytaniu ze zrozumieniem u osób ze spektrum autyzmu}

    Rozumienie języka, zarówno mówionego, jak i pisanego, stanowi złożone zadanie, na które składa się wiele różnych umiejętności i procesów poznawczych (\cite{cain2008children}). Tak zwany prosty model czytania (\emph{Simple View of Reading}, \cite{hoover1990simple}) wyróżnia dwa komponenty: dekodowanie i rozumienie języka. 

\section{Technologie informacyjno-komunikacyjne w terapii dzieci ze spektrum autyzmu}
\chapter{Projekt gry}
\thispagestyle{firststyle}

\section{Planowane efekty}

\section{Projekt ćwiczeń}

\section{User experience}
\chapter{Implementacja i ewaluacja}
\thispagestyle{firststyle}

\section{Technologia}

\section{User experience}

\section{Efekty terapeutyczne}
\chapter{Zakończenie}
\thispagestyle{firststyle}

\section{Wnioski}

\section{Wyzwania}

\renewcommand{\bibsetup}{\thispagestyle{firststyle}}
{\linespread{1.25}\printbibliography[title=Bibliografia, heading=bibintoc]}

\end{document}
