\addcontentsline{toc}{chapter}{Abstract}
% \thispagestyle{firststyle}
{\centering\large
    % \vspace{0.5cm}
    % Jagiellonian University\\
    {\Huge\emph{Abstract}\\}
    \vspace{0.5cm}
    % Faculty of Philosophy\\Institute of Philosophy\\
    % \vspace{1cm}
    % \textbf{Information and Communication Technology in Therapy for Children with Autism Spectrum Disorder: Game Improving Reading Comprehension\\}
    % \vspace{0.5cm}
    % \authorname\\
    % \vspace{0.5cm}
}

This thesis contains a description of the project and the implementation of a computer game for children with autism spectrum disorder improving reading comprehension, that allows for ongoing monitoring of the progress of a player and individualising the content of the game.
Language skills, especially language comprehension, are one of the areas in which people with ASD have difficulties.
Information and communication technology is often used in education and therapy for people with ASD due to the possibility of providing predictable environment and learning in controlled conditions.

\vspace{1cm}
\noindent\textbf{Keywords:}
autism spectrum disorder, information and communication technology, reading comprehension