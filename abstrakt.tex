% \addcontentsline{toc}{chapter}{Abstrakt}
\thispagestyle{firststyle}
{\centering\large
    % \vspace{0.5cm}
    % Uniwersytet Jagielloński\\
    {\Huge\emph{Abstrakt}\\}
    \vspace{0.5cm}
    % Wydział Filozoficzny\\Instytut Filozofii\\
    % \vspace{1cm}
    % \textbf{Technologie informacyjno-komunikacyjne w terapii dzieci\\z zaburzeniami ze spektrum autyzmu: gra wspomagająca rozwój umiejętności czytania ze zrozumieniem\\}
    % \vspace{0.5cm}
    % \authorname\\
    % \vspace{0.5cm}
}

\sloppy
Technologie informacyjno-komunikacyjne stanowią narzędzie często wykorzystywane w~edukacji i terapii osób z zaburzeniami ze spektrum autyzmu, z uwagi na łatwość zapewnienia przewidywalnego środowiska i nauki w kontrolowanych warunkach.
Jednym z obszarów, w~ramach których u osób z ASD można zaobserwować trudności, są kompetencje językowe, w~szczególności rozumienie języka.
Niniejsza praca zawiera opis projektu i implementacji gry komputerowej dla dzieci z ASD wspomagającej ćwiczenie umiejętności czytania ze zrozumieniem, pozwalającej na bieżące monitorowanie postępów gracza oraz zindywidualizowanie treści zadań.

\vspace{1cm}
\noindent\textbf{Słowa kluczowe:}
zaburzenia ze spektrum autyzmu, technologie informacyjno-komunikacyjne, czytanie ze zrozumieniem

% \addcontentsline{toc}{chapter}{Abstract}
% \thispagestyle{firststyle}
{\centering\large
    % \vspace{0.5cm}
    % Jagiellonian University\\
    {\Huge\emph{Abstract}\\}
    \vspace{0.5cm}
    % Faculty of Philosophy\\Institute of Philosophy\\
    % \vspace{1cm}
    % \textbf{Information and Communication Technology in Therapy for Children with Autism Spectrum Disorder: Game Improving Reading Comprehension\\}
    % \vspace{0.5cm}
    % \authorname\\
    % \vspace{0.5cm}
}

Information and communication technology is often used in education and therapy for people with autism spectrum disorder due to the possibility of providing predictable environment and learning in controlled conditions.
One of the areas in which people with ASD may have difficulties are language skills, especially language comprehension.
This thesis contains a~description of the project and the implementation of a computer game for children with ASD improving reading comprehension, that allows for ongoing monitoring of the progress of a~player and individualisation of the content of exercises.

\vspace{1cm}
\noindent\textbf{Keywords:}
autism spectrum disorder, information and communication technology, reading comprehension